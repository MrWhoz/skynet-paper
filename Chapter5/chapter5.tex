\chapter{Tổng kết và Hướng phát triển cho tương lai}

% **************************** Define Graphics Path **************************
\ifpdf
    \graphicspath{{Chapter5/Figs/Raster/}{Chapter5/Figs/PDF/}{Chapter5/Figs/}}
\else
    \graphicspath{{Chapter5/Figs/Vector/}{Chapter5/Figs/}}
\fi

\section{Tổng kết}
\subsection{Kết quả đạt được}

\subsubsection*{Sensor Box}
\begin{itemize}
\item[•] Tìm hiểu được về các thành phần trong khí thải và mức độ ảnh hưởng tới môi trường.
\item[•] Tìm hiểu các loại cảm biến và cách thức hoạt động. 
\item[•] Hiện thực hoàn chỉnh các chức năng module thu thập dữ liệu từ các cảm biến.
\item[•] Thiết kế và hiện thực được Sensor Box có khả năng hoạt động dưới những môi trường nắng, mưa, gió, bụi.
\item[•] Nghiên cứu và tích hợp được thêm module sử dụng năng lượng mặt trời và mạch nạp pin cho phép hệ thống hoạt động độc lập với điện lưới.
\item[•] Thu thập được dữ liệu từ 3 Sensor Node đã hiện thực tại những địa điểm khác nhau trong suốt thời gian gần 2 tháng.
\end{itemize}

\subsubsection*{Server}
\begin{itemize}
\item[•] Cung cấp API và API log cho nhà phát triển.
\item[•] Áp dụng mô hình kiến trúc hệ thống thuận tiện cho phát triển ứng dụng realtime.
\item[•] Xây dựng giao diện trực quan và thân thiện với người dùng, liên kết cơ sở dữ liệu thời gian thực cung cấp thông tin kịp thời để tiện lợi cho quá trình theo dõi và đánh giá.
\item[•] Đồ thị Google Maps và đồ thị dữ liệu cho việc theo dõi các thông số.
\item[•] Giao tiếp thông qua gmail giữa người dùng và người quản lý.
\end{itemize}

\subsubsection*{Ứng dụng thiết bị di động}
\begin{itemize}
\item[•] Sử dụng Framework Ionic xây dựng được ứng dụng di động trên nhiều nền tảng khác nhau.
\item[•] Liên kết ứng dụng với API được cung cấp, thực hiện một số chức năng cơ bản: hiên thị thông tin node, đồ thị, thống kê.
\end{itemize}

\subsection{Những thuận lợi,khó khăn và mặt hạn chế}
\subsubsection*{Thuận lợi}
\begin{itemize}
\item[•] Giáo viên hướng dẫn thường xuyên lên kế hoạch báo cáo 2 tuần một lần để cập nhật tình hình và giúp đỡ nhóm giải quyết những vấn đề khó khăn trong suốt quá trình làm đề tài luận văn.
\item[•] Được hỗ trợ kinh phí từ giáo viên hướng dẫn.
\item[•] Thời tiết nắng, mưa, gió, bụi đầy đủ để có thể thử nghiệm độ hiểu quả của Sensor Box.
\item[•] Được phòng Lab BKIT hỗ trợ về mặt máy móc, kỹ thuật trong suốt quá trình làm luận văn tốt nghiệp.
\end{itemize}

\subsubsection*{Khó khăn}
\begin{itemize}
\item[•] Các module năng lượng mặt trời không có sẵn, phải tìm các module khác để thay thế.
\item[•] Chưa có thiết bị chuẩn để thực hiện quá trình calibration cho các cảm biến.
\item[•] Dành nhiều thời gian để tiếp cận với hướng lập trình phần mềm và công nghệ mới.
\item[•] Thử nghiệm nhiều với các module sim khác nhau.
\item[•] Thiếu các thiết bị di động để thử nghiệm tính đa nền tảng của ứng dụng di động trên các nền tảng iOS, Windows Phone.
\item[•] Dành nhiều thời gian tìm hiểu và làm quen với các công nghệ mới.
\end{itemize}
\subsubsection*{Mặt hạn chế của đề tài luận văn}
\begin{itemize}
\item[•] Việc tích hợp module năng lượng mặt trời và mạch nạp pin chưa thật sự hiệu quả để Sensor Box hoạt động ổn định độc lập với điện lưới, vấn đề được khắc phục trong tương lai nhờ tìm các tấm năng lượng mặt trời khác hoặc các nguồn năng lượng ngoài như: gió, mưa.
\item[•] Chỉ dừng ở mức thu thập và thống kê, chưa phát triển phân tích các thông số môi trường.
\item[•] Cụm server đặt trên Raspberry Pi 3 không đủ sức mạnh đáp ứng số lượng lớn người dùng cùng một lúc.
\item[•] Vẫn còn một số chức năng chưa hoàn thiện: quản lý người dùng cho ứng dụng di động.
\item[•] Chưa hiện thực cơ chế bảo mật mã hóa gói tin.
\item[•] Quá trình calibration chưa được tốt vì chưa có thiết bị chuẩn.
\end{itemize}


\section{Những kiến thức tiếp thu được qua đề tài}
\subsection{Về mặt kiến thức tổng quát}
\begin{itemize}
	\item[•] Nắm bắt được mô hình xây dựng ứng dụng IoT và các hướng phát triển.
	\item[•] Tiếp thu được kiến thức về môi trường và các yếu tố ảnh hưởng.
	\item[•] Làm quen với các chuẩn giao tiếp mới.
	\item[•] Các thức triển khai và thực nghiệm hệ thống.
	\item[•] Củng cố và phát triển khả năng viết và cấu trúc bản báo cáo. 
\end{itemize}
\subsection{Về mặt hiện thực thiết bị}
\begin{itemize}
	\item[•] Ứng dụng tốt việc lập trình với module SIM800L.
	\item[•] Biết cách áp dụng được nguồn pin năng lượng mặt trời vào thiết bị hoạt động thực tế.
	\item[•] Nắm bắt và xử lý được nhiều trường hợp trong quá trình xây dựng Sensor Box.
\end{itemize}
\subsection{Về mặt xây dựng hệ thống Server và ứng dụng}
\begin{itemize}
	\item[•] Mô tả được mô hình cụm Server và phát triển các ứng dụng Web và di động.
	\item[•] Làm quen với thiết kế và xử lý trong hệ cơ sở dữ liệu.
	\item[•] Tìm tòi học hỏi và áp dụng các công nghệ xử lý thời gian thực.
	\item[•] Cải thiện kỹ năng lập trình và thiết kế trong mảng UI(giao diện người dùng) và UX(trải nghiệm người dùng).
	\item[•] Nắm được cách thức phát triển và áp dụng các API.
\end{itemize}
\section{Hướng phát triển tương lai}
\begin{itemize}
	%sensor node here
\item[•] Chuẩn hóa các cảm biến MQ07, MQ135 bằng các thiết bị đo chuẩn để lấy được giá trị chính xác nhất có thể.
\item[•] Tiếp tục hoàn thiện phần Server hoạt động hiệu quả, đáp ứng hoạt động cho một số lượng lớn các node cảm biến.
\item[•] Xây dựng hệ thống quản lý node theo người dùng, đáp ứng nhu cầu quản lý và thống kê theo từng người.
\item[•] Xây dựng thuật toán phân tích dữ liệu môi trường và áp dụng Machine learning trong việc phân tích dữ liệu để đưa ra kết quả dự đoán tính trạng ô nhiễm môi trường và mật độ giao thông.
\item[•] Nghiên cứu và ứng dụng các nguồn thu năng lượng từ gió, mưa cho Sensor Node.
\item[•] Phát triển ứng dụng thiết lập, chỉnh sửa, thay thế Sensor Node dành cho người phát triển thiết bị.
\item[•] Tích hợp module bluetooth để thu thập các dữ liệu người dùng xung quanh Sensor Node.
\item[•] Tích hợp công nghệ nạp và nâng cấp firmware không dây.
\end{itemize}
