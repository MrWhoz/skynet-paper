\chapter{Tổng kết và Hướng phát triển cho tương lai}

% **************************** Define Graphics Path **************************
\ifpdf
    \graphicspath{{Chapter5/Figs/Raster/}{Chapter5/Figs/PDF/}{Chapter5/Figs/}}
\else
    \graphicspath{{Chapter5/Figs/Vector/}{Chapter5/Figs/}}
\fi

\section{Tổng kết}
\subsection{Những gì đạt được}

\subsubsection*{Server}
\begin{itemize}
\item[•]Cung cấp API và API log cho nhà phát triển.
\item[•]Áp dụng mô hình kiến trúc hệ thống thuận tiện cho phát triển ứng dụng realtime.
\item[•]Xây dựng giao diện trực quan và thân thiện với người dùng, liên kết cơ sở dữ liệu thời gian thực cung cấp thông tin kịp thời để tiện lợi cho quá trình theo dõi và đánh giá.
\item[•]Đồ thị cho việc theo dõi các thông số.
\item[•]Giao tiếp thông qua gmail giữa người dùng và người quản lý.
\end{itemize}


\subsubsection*{Ứng dụng thiết bị di động}
\begin{itemize}
\item[•] Sử dụng Framework Ionic xây dựng được ứng dụng di động trên nhiều nền tảng khác nhau.
\item[•] Liên kết ứng dụng với API được cung cấp, thực hiện một số chức năng cơ bản: hiên thị thông tin node, đồ thị…
\end{itemize}

\subsection{Những khó khăn và hạn chế}

\begin{itemize}
\item[•] Các thiết bị module không có sẵn, phải tìm các thiết bị khác thay thế.
\item[•] Dành nhiều thời gian để tiếp cận với hướng lập trình phần mềm và công nghệ mới.
\item[•] Dữ liệu xử lý lớn, server cần phải đáp ứng được vấn đề thời gian thực và tránh xảy ra lỗi.
\item[•] Chỉ dừng ở mức thu thập và thống kê, chưa phát triển phân tích các thông số môi trường.
\item[•] Vẫn còn một số chức năng chưa hoàn thiện: quản lý người dùng cho Mobile app.
\item[•] Cụm server đặt trên Raspberry Pi 3 khổng đủ sức mạnh đáp ứng số lượng lớn người dùng cùng một lúc.
\end{itemize}
\section{Những kiến thức tiếp thu được qua đề tài}
\subsection{Về mặt kiến thức tổng quát}
\begin{itemize}
	\item[•] Nắm bắt được mô hình xây dựng ứng dụng IoT và các hướng phát triển.
	\item[•] Tiếp thu được kiến thức về môi trường và các yếu tố ảnh hưởng.
	\item[•] Làm quen với các chuẩn giao tiếp mới.
	\item[•] Các thức triển khai và thực nghiệm hệ thống.
	\item[•] Củng cố và phát triển khả năng viết và cấu trúc bản báo cáo.
\end{itemize}
\subsection{Về mặt hiện thực thiết bị}
\begin{itemize}
	\item[•] Ứng dụng lập trình với module SIM800L.
	\item[•] Biết cách áp dụng được nguồn pin năng lượng mặt trời vào thiết bị hoạt động thực tế.
	\item[•] Nắm bắt và xử lý được nhiều trường hợp trong xây dựng Sensor Box.
\end{itemize}
\subsection{Về mặt xây dựng hệ thống Server và ứng dụng}
\begin{itemize}
	\item[•] Mô tả được mô hình cụm Server và phát triển các ứng dụng Web và di động.
	\item[•] Làm quen với thiết kế và xử lý trong hệ cơ sở dữ liệu.
	\item[•] Tìm tòi học hỏi và áp dụng các công nghệ xử lý thời gian thực.
	\item[•] Cải thiện kỹ năng lập trình và thiết kế trong mảng UI(giao diện người dùng) và UX(trải nghiệm người dùng).
	\item[•] Nắm được cách thức phát triển và áp dụng các API.
\end{itemize}
\section{Hướng phát triển tương lai}
\begin{itemize}
	%sensor node here
\item[•]Tiếp tục hoàn thiện phần Server hoạt động hiệu quả, đáp ứng hoạt động cho một số lượng lớn các node cảm biến.
\item[•]Xây dựng hệ thống quản lý node theo người dùng, đáp ứng nhu cầu quản lý và thống kê theo từng người.
\item[•]Xây dựng thuật toán phân tích dữ liệu môi trường và áp dụng Machine learning trong việc phân tích dữ liệu để đưa ra kết quả dự đoán tính trạng ô nhiễm môi trường và mật độ giao thông.
\item[•]Hoàn thành các chức năng cần thiết cho Mobile app.
\end{itemize}
