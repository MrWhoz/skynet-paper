% **************************** Define Graphics Path **************************
\ifpdf
\graphicspath{{Chapter3/Figs/Raster/}{Chapter3/Figs/PDF/}{Chapter3/Figs/}{Chapter3/Figs/Web/}{Chapter3/Figs/Server/}{Chapter3/Figs/Mobile/}}
\else
\graphicspath{{Chapter3/Figs/Vector/}{Chapter3/Figs/}}
\fi

\chapter{THIẾT KẾ HỆ THỐNG}\label{chap: design}
Áp dụng những kiến thức tại \textbf{Chương 2: Cơ sở lý thuyết}, chương 3 đưa ra những thiết kế về cấu trúc hệ thống, mô hình sensor box và APIs được hỗ trợ.
\section{Kiến trúc mô hình hệ thống}\label{sec:struc}
Như đã giới thiệu một số mô hình IoT thực tế ở Mục \ref{sec:xuhuongiot}, mô hình IoT không dùng Gateway được chúng tôi áp dụng cho hệ thống. Hệ thống bao gồm các thành phần chính như sau: các node cảm biến, trung tâm lưu trữ dữ liệu và ứng dụng truy xuất và hiển thị thông tin dữ liệu.
\begin{figure}[H]
	\centering    
	\includegraphics[width=0.7\textwidth]{model}
	\caption[Kiến trúc mô hình hệ thống]{Kiến trúc mô hình hệ thống}
	\label{fig:system}
\end{figure}

\section{Các node cảm biến}\label{sec:cacloaicambien}
Các node cảm biến: là một thiết bị được lắp đặt trên những đoạn đường hoặc các địa điểm thích hợp cần để lấy dữ liệu tại khu vực đó, nhiệm vụ chủ yếu là lấy dữ liệu từ các cảm biến đo được sau đó gửi dữ liệu đó lên máy chủ bằng module Sim GPRS.
Để các node cảm biến này được hoạt động dưới tác nhân của môi trường như gió, mưa, bụi, và hơn hết có thể tận dụng nguồn năng lượng mặt trời để nạp pin cho mạch hoạt động độc lập với điện lưới, chúng tôi đã đề xuất mô hình thiết kế node cảm biến để có thể đáp ứng các yêu cầu như sau:

\begin{wrapfigure}{l}[4pt]{0.4\linewidth}
    \includegraphics[scale=0.3]{house_home}
    \caption[Mô hình căn nhà 3D]{Mô hình căn nhà 3D}
    \label{fig:house_home}
\end{wrapfigure}
Lấy ý tưởng từ căn nhà gồm 4 bức tường và 2 mái che như Hình \ref{fig:house_home}, tấm năng lượng mặt trời được thiết kế và gắn như mái che của căn nhà, còn các mạch điện và cảm biến sẽ được đặt bên trong ngôi nhà xung quanh các bức tường được làm từ nhựa mica. Việc lắp đặt node cảm biến này theo hướng các tấm năng lượng mặt trời 1 phía hướng về đông và 1 phía hướng về tây nhằm giúp tấm năng lượng có thể lấy được năng lượng tối đa.

Từ kết quả tìm hiểu các loại khí thải trong mục \ref{sec:yeuto_khithai} đã được nên kết hợp với tình trạng các loại cảm biến hiện có trên thị trường hiện này thì nhóm chúng tôi đã quyết định đưa ra các loại cảm biến cần thiết cho một node cảm biến như sau:



%\\\\\\\\\\\\\\\\\\\\\\\\\\\\\\\\\\\\\\GP2\\\\\\\\\\\\\\\\\\\\\\\\\\\\\\\\\\\\\\\\\\\\%
\subsection{Cảm biến bụi GP2} 
Cảm biến bụi GP2 là một bộ cảm biến chất lượng không khí quang học, được thiết kế đo những hạt bụi. Cảm biến được thiết kế một điốt phát hồng ngoại và một photostransistor được sắp xếp thành đường chéo thiết bị này, cho phép nó phát hiện ánh sáng phản xạ của bụi trong không khí.  Nó đặc biệt hiệu quả trong việc phát hiện các hạt rất mịn như khói thuốc lá, và thường được sử dụng trong các hệ thống lọc không khí.

Cảm biến bụi GP2 tiêu tốn dòng rất ít (20 mA cao nhất, 11mA chế độ chạy thường), có thể hỗ trợ nguồn cung cấp lên tới 7VDC. Tín hiệu đầu ra của cảm biến là dạng tín hiệu tương tự dùng để đo mật độ bụi, với độ nhạy $0.5V/0.1mg/{m}^{3}$.

Thông số kỹ thuật:
\begin{itemize}
\item[•]Nguồn hoạt động: 5VDC
\item[•]Dòng tiêu thụ: 10mA
\item[•]Ngõ ra analog
\item[•]Nhiệt độ hoạt động: $-40^{o}$ $-$ $85 ^{o}$ C
\end{itemize}

\begin{figure}[H]
	\centering    
	\includegraphics[width=0.5\textwidth]{gp2}
	\caption[Cảm biến bụi GP2 ]{Cảm biến bụi GP2}
	\label{fig:gp2}
\end{figure}

%\\\\\\\\\\\\\\\\\\\\\\\\\\\\\\\\\\\\\\	MQ135	\\\\\\\\\\\\\\\\\\\\\\\\\\\\\\\\\\\\\\\\\\\\%
\newpage
\subsection{Cảm biến chất lượng không khí MQ135} 
Cảm biến chất lượng không khí MQ135 thường được dùng trong các thiết bị kiểm tra chất lượng không khí bên trong cao ốc, văn phòng, thích hợp để phát hiện khí NH3, Nox, Ancol, Benzen, khói và khí CO2.Cảm biến này với độ nhạy cao và thời gian đáp ứng nhanh. Tín hiệu ngõ ra dạng analog và digital. Cảm biến này có thể hoạt động ở nhiệt độ từ khoảng: $-10^{o}$C đến $50^{o}$C và tiêu thụ dòng khoảng 300mA tại 5V.
\begin{figure}[H]
	\centering    
	\includegraphics[width=0.5\textwidth]{mq135}
	\caption[Cảm biến MQ135]{Cảm biến MQ135}
	\label{fig:mq135}
\end{figure}
Thông số kỹ thuật:
\begin{itemize}
\item[•]Điện áp cung cấp: <=24 VDC
\item[•]Điện áp heater: 5V AC/DC
\item[•]Sử dụng chip so sánh LM393c.
\item[•]Hai tín hiệu đầu ra (digital và analog)
\item[•]Tín hiệu analog từ 0 đến 5V.
\item[•]Dải phát hiện từ 10 đến 1000ppm
\end{itemize}

\begin{figure}[H]
	\centering    
	\includegraphics[width=0.7\textwidth]{mq135_mqh1}
	\caption[Giản đồ chỉ sự biến đổi của Rs/Ro và giá trị ppm của MQ135]{Giản đồ chỉ sự biến đổi của Rs/Ro và giá trị ppm của MQ135}
	\label{fig:mq135_mqh1}
\end{figure}

\begin{figure}[H]
	\centering    
	\includegraphics[width=0.7\textwidth]{mq135_mqh2}
	\caption[Giản đồ chỉ sự biến đổi của Rs/Ro đối với nhiệt độ, độ ẩm]{Giản đồ chỉ sự biến đổi của Rs/Ro đối với nhiệt độ, độ ẩm}
	\label{fig:mq135_mqh2}
\end{figure}


Giá trị chất lượng không khí(ppm) được xác định theo biểu đồ Hình \ref{fig:mq135_mqh1} giá trị này thay đổi theo sự biến đổi của giá trị Rs/Ro, theo Hình \ref{fig:mq135_mqh2} thì giá trị Rs/Ro bị ảnh hưởng bởi nhiệt độ và độ ẩm.

Để xác định được giá trị Rs/Ro từ ngõ ra analog của mạch cảm biến MQ-135 ta có công thức như sau:
\begin{center}
$\displaystyle \frac{R_{S}}{R_{L}} = \frac{V_{C}-V_{RL}}{V_{RL}}$
\end{center}

Giá trị Rs/Ro này được vi xử lý tính toán dựa trên đầu ra analog của cảm biến MQ-135 và được đưa vào thư viện mq135.h để có thể lấy được giá trị chất lượng không khí ppm theo sự thay đổi của nhiệt độ và độ ẩm của môi trường thực tế.

Thư viện mq135.h cung cấp cho chúng ta các hàm toán học đã được nhà sản xuất tính toán sẵn để có thể lấy được giá trị chất lượng không khí ppm dựa vào thay đổi của môi trường nhiệt độ và độ ẩm.



%\\\\\\\\\\\\\\\\\\\\\\\\\\\\\\\\\\\\\\	MQ07	\\\\\\\\\\\\\\\\\\\\\\\\\\\\\\\\\\\\\\\\\\\\%
\subsection{Cảm biến chất lượng không khí MQ07} 
Cảm biến khí CO MQ-7 có thể pháp hiện khí CO tập trung ở những nơi khác nhau từ 10 đến 1000ppm. Cám biến này với độ nhạy cao và thời gian đáp ứng nhanh. Tín hiệu ngõ ra dạng analog và digital. Cảm biến này có thể hoạt động ở nhiệt độ từ khoảng: -10C đến 50C và tiêu thụ dòng khoảng 250mA tại 5V.
\begin{figure}[H]
\centering    
\includegraphics[width=0.6\textwidth]{mq07}
\caption[Cảm biến MQ07]{Cảm biến MQ07}
\label{fig:mq07}
\end{figure}
Thông số kỹ thuật:
\begin{itemize}
\item[•]Điện áp cung cấp: từ 3 đến 5VDC.
\item[•]Sử dụng chip so sánh LM393 và MQ-7.
\item[•]Hai tín hiệu đầu ra (digital và analog)
\item[•]Tín hiệu analog từ 0 đến 5V 
\item[•]Dải phát hiện từ 10 đến 1000ppm
\item[•]Kích thước: 33 x 20 x 16mm.
\end{itemize}
\begin{figure}[H]
\centering    
\includegraphics[width=0.7\textwidth]{mq07_mqh1}
\caption[Giản đồ chỉ sự biến đổi của Rs/Ro và giá trị ppm của MQ07]{Giản đồ chỉ sự biến đổi của Rs/Ro và giá trị ppm của MQ07}
\label{fig:mq07_mqh1}
\end{figure}


\begin{figure}[H]
\centering    
\includegraphics[width=0.7\textwidth]{mq07_mqh2}
\caption[Giản đồ chỉ sự biến đổi của Rs/Ro đối với nhiệt độ, độ ẩm]{Giản đồ chỉ sự biến đổi của Rs/Ro đối với nhiệt độ, độ ẩm}
\label{fig:mq07_mqh2}
\end{figure}


Giá trị chất lượng không khí(ppm) được xác định theo biểu đồ Hình \ref{fig:mq07_mqh1} giá trị này thay đổi theo sự biến đổi của giá trị Rs/Ro, theo Hình \ref{fig:mq07_mqh2} thì giá trị Rs/Ro bị ảnh hưởng bởi nhiệt độ và độ ẩm.

Để xác định được giá trị Rs/Ro từ ngõ ra analog của mạch cảm biến MQ-7 ta có công thức như sau:
\begin{center}
$\displaystyle \frac{R_{S}}{R_{L}} = \frac{V_{C}-V_{RL}}{V_{RL}}$
\end{center}

Giá trị Rs/Ro này được vi xử lý tính toán dựa trên đầu ra analog của cảm biến MQ-7 và được đưa vào thư viện mq07.h để có thể lấy được giá trị chất lượng không khí ppm theo sự thay đổi của nhiệt độ và độ ẩm của môi trường thực tế.

Thư viện mq07.h cung cấp cho chúng ta các hàm toán học đã được nhà sản xuất tính toán sẵn để có thể lấy được giá trị chất lượng không khí ppm dựa vào thay đổi của môi trường nhiệt độ và độ ẩm.



%\\\\\\\\\\\\\\\\\\\\\\\\\\\\\\\\\\\\\\	Cảm biến nhiệt độ DS18B20 IC	\\\\\\\\\\\\\\\\\\\\\\\\\\\\\\\\\\\\\\\\\\\\%
\subsection{Cảm biến nhiệt độ DS18B20 IC} 
Cảm biến nhiệt độ DS18B20 là cảm biến loại tín hiệu đầu ra digital đo nhiệt độ mới của hãng MAXIM với độ phân giải cao (12bit). IC được sử dụng giao tiếp 1 dây rất gọn gàng, dễ lập trình và giao tiếp nhiều DS18B20 trên cùng 1 dây. IC còn có chức năng cảnh báo nhiệt độ khi vượt ngưỡng và đặc biệt hơn là có thể cấp nguốn từ chân data.

\begin{figure}[H]
	\centering    
	\includegraphics[width=0.7\textwidth]{ds18b20}
	\caption[Cảm biến nhiệt độ DS18B20]{Cảm biến nhiệt độ DS18B20}
	\label{fig:ds18b20}
\end{figure}

Thông số kỹ thuật:
\begin{itemize}
\item[•]Nguồn: 3-5.5V
\item[•]Dải đo nhiệt độ: $-55 -125_{o}C$
\item[•]Sai số: $+-0.5_{o}C$
\item[•]Độ phân giải: 9-12bits
\item[•]Thời gian chuyển đổi nhiệt độ: 750ms
\end{itemize}
Sơ đồ kết nối IC BS18B20 để lấy được tín hiệu data
\begin{figure}[H]
	\centering    
	\includegraphics[width=0.6\textwidth]{ds18b20_ketnoi}
	\caption[Sơ đồ kết nối DS18B20]{Sơ đồ kết nối DS18B20}
	\label{fig: ds18b20_ketnoi}
\end{figure}
Chân Digital input sẽ được đưa tới chân Digital input của board arduino nano, hiện tại nhà sản xuất cũng cung cấp cho lập trình viên về bộ thư viện để có thể sử dụng trong việc phát triển.

\newpage


\section{Chuẩn giao tiếp giữa các node tới máy chủ}
Để các node cảm biến gửi dữ liệu lên máy chủ thì chúng tôi đã sử dụng module Sim800L được giới thiệu trong mục \ref{sec:sim800l}. Quá trình kết nối của các node cảm biến tới máy chủ thông qua giao thức TCP được thực hiện dựa trên mô hình máy trạng thái như sau:


\begin{figure}[H]
\centering   
\includegraphics[width=1\textwidth]{sim800_status}
\caption[Sơ đồ trạng thái GPRS cho đơn kết nối]{Sơ đồ trạng thái GPRS cho đơn kết nối}
\label{fig:sim800_status}
\end{figure}

Để gửi được dữ liệu đến máy chủ cần phải thiết lập được kết nối TCP từ module SIM800 đến máy chủ thông qua 11 trạng thái như Hình \ref{fig:sim800_status}, các trạng thái được chuyển đổi thông qua các tập lệnh AT đã được đề cập tại mục \ref{sec:sim800l}. Đến trạng thái 7 (CONNET OK) chúng ta đã giữ được kết nối TCP tới máy chủ và tại trạng thái này dùng lệnh AT+CIPSEND để thực hiện việc gửi gói dữ liệu lên cho máy chủ.

Gói dữ liệu được gửi theo phương thức GET, và gói tin được gửi lên máy chủ bao gồm:
\begin{table}[H]
\centering
\label{table:sensornode_api}
\begin{tabular}{|l|l|l|l|l|l|}
\hline
\begin{tabular}[c]{@{}l@{}}NODE\\ ID\end{tabular} & \begin{tabular}[c]{@{}l@{}}Giá trị CO\\ (ppm)\end{tabular} & \begin{tabular}[c]{@{}l@{}}Giá trị \\ nhiệt độ \\ (Độ C)\end{tabular} & \begin{tabular}[c]{@{}l@{}}Giá trị\\ độ bụi\\ (ppm)\end{tabular} & \begin{tabular}[c]{@{}l@{}}Giá trị chất\\ lượng không \\ khí (ppm)\end{tabular} & \begin{tabular}[c]{@{}l@{}}Giá trị dung\\ lượng pin (\%)\end{tabular} \\ \hline
\end{tabular}
\end{table}
\section{Thiết kế API}\label{sec: api}
Mục tiêu đề tài không chỉ thu thập và thống kê mà còn chia sẻ dữ liệu. Để làm được điều đó thì cần phải có API cung cấp cho những nhà phát triển thứ 3, có thể dùng API được cung cấp để sử dụng dữ liệu để phát triển. Bởi vì vậy hệ thống có xây dựng hệ thống API mở.

Ở bảng \ref{table: apilist}, hệ thống cung cấp các API về quản lý và truy vấn dữ liệu của các sensor node. Sử dụng hai method POST và GET để gửi request tới Server. 

\begin{table}[H]
	\centering
	\caption{Bảng API tương tác với dữ liệu về các node}
	\begin{tabular}{|l|l|l|l|}
		\hline
		Method & URL            & Miêu tả         & Yêu cầu xác thực người dùng        \\ \hline
		POST   & /node/initnew       & Khởi tạo node mới                & Có           \\ \hline
		POST   & /node/updatenode     & Cập nhật thông tin node & Có \\ \hline
		POST   & /node/replacenode       & Thay thế node  & Có       \\ \hline
		GET   & /node/pushdata & Push dữ liệu               &         Không                \\ \hline
		GET   & /node/getdata    & Get dữ liệu thu thập của node         &  Không      \\ \hline
		GET   & /node/getinfo   & Get thông tin của node & Không  \\ \hline
	\end{tabular}
	\label{table: apilist}
\end{table}

Giao thức được lựa chọn sử dụng trong việc push data là giao thức HTTP mà không phải các giao thức thuần để phát triển IoT như MQTT hay CoAp vì tính phổ biến và dễ thiết lập sử dụng. Chỉ cần thiết lập chuỗi URL có đính kèm thông tin là có thể hòa nhập với hệ thống mà không cần thiết lập gì thêm, có thể ví như là "plug and play" vì tính đơn giản và hoạt động với mọi thiết bị có hỗ trợ kết nối Internet.

Bù lại, so với các giao thức tối ưu dung lượng gói tin cho phát triển IoT như MQTT, gói tin gửi qua HTTP có dung lượng nhiều hơn đáng kể nhưng với tần số hoạt động truyền dữ liệu của đề tài thì vẫn có thể chấp nhận được và không tạo ra sự khác biệt lớn khi hoạt động.

Về phần API trợ giúp đăng ký và xác thực người dùng được quy ước như bảng \ref{table: apiuser}.
\begin{table}[H]
	\centering
	\caption{Bảng API tính năng đối với người dùng}
	\begin{tabular}{|l|l|l|}
		\hline
		Method & URL            & Miêu tả                \\ \hline
		POST   & /auth/login       & Đăng nhập                           \\ \hline
		GET   & /auth/logout   & Đăng xuất   \\ \hline
		POST   & /user/register     & Đăng ký người dùng mới \\ \hline
	\end{tabular}
	\label{table: apiuser}
\end{table}
\subsection{Cách thức tương tác cập nhật dữ liệu}

Trong việc tương tác cập nhật dữ liệu, hiện nay mô hình truy thường được sử dụng là \textbf{polling}. Client và server phải liên tục gửi các request liên tục cách nhau trong một thời gian cố định để kiểm tra liệu có sự thay đổi cập nhật dữ liệu. 
\begin{figure}[H]
	\centering    
	\includegraphics[width=1.0\textwidth]{polling}
	\caption[Mô hình tương tác polling]{Mô hình tương tác polling}
	\label{fig: polling}
\end{figure}
Phương pháp này dễ hiện thực và áp dụng, tuy nhiên sẽ mắc nhược điểm là tốn nhiều gói tin request vô dụng làm giảm hiệu năng hệ thống, và không đáp ứng tính realtime nếu như chu kì gửi request dài.



Vì hiện tại nhóm chúng tôi đang hướng tới hướng phát triển IoT, đòi hỏi yêu cầu về tính realtime cũng như tối ưu hóa hệ thống. Và để đáp ứng yêu cầu đó, nhóm đã tìm tới giải pháp giữ socket kết nối từ \textbf{Database} (RethinkDB) <-> \textbf{Server} (Chạy trên nền tảng Nodejs) <-> \textbf{Client} (web browser)

\begin{figure}[H]
	\centering    
	\includegraphics[width=1.0\textwidth]{realtime}
	\caption[Mô hình tương tác realtime dựa trên Socket]{Mô hình tương tác realtime dựa trên Socket}
	\label{fig: realtime}
\end{figure}

Như hình \ref{fig: realtime}, mô hình hoạt động đơn giản hơn rất nhiều, giảm bớt số lượng request không đáng có và đảm bảo tính realtime cho hệ thống.

Một điểm mạnh ở mô hình này nữa là có thể phát triển nhiều server hoặc ứng dụng di động sử dụng chung một hệ quản lý cơ sở dữ liệu nhưng vẫn đảm bảo sự đồng bộ dữ liệu cập nhật tới người dùng đầu cuối theo thời gian thực.
\begin{figure}[H]
	\centering    
	\includegraphics[width=1.0\textwidth]{multiserver}
	\caption[Mô hình tương tác cập nhật nhiều Server]{Mô hình tương tác cập nhật nhiều Server và Client}
	\label{fig: multiserver}
\end{figure}
\section{Mô hình ứng dụng trình bày dữ liệu}
\subsection*{Chức năng Web Server}
Đối với người dùng:
\begin{itemize}
\item[•] Cung cấp cho người dùng thông tin về nơi chứa node cảm biến.
\item[•] Người dùng có thể xem biểu đồ dữ liệu thông số môi trường tại vị trí tương ứng với node cảm biến đã được thiết lập.
\item[•] Xem được tình trạng và biểu đồ thống kê của các node cảm biến đang hoạt động.
\item[•] Gửi thông tin phản hồi về cho người quản lý thông qua email.
\item[•] Cung cấp thông tin lịch sử các hoạt động sử dung API, giúp cho bên thứ 3 có thể dễ dàng phát triển.
\end{itemize}

Đối với người quản lý:
\begin{itemize}
\item[•] Bao gồm tất cả chức năng của người dùng bình thường.
\item[•] Quản lý chỉnh sửa thông tin các node cảm biến: thêm, xóa, cập nhật, thay đổi.
\end{itemize}

\subsection*{Chức năng Ứng dụng di động}
Hiện thị thông tin các node cảm biến đang hoạt động qua màn hình chính, và biểu đồ dữ liệu của từng node cảm biến theo từng ngày, hiện thị đồ thị (2 kiểu đồ thị là line chart và bar chart).

\subsection*{Biểu đồ High level Usecase của Web Server}
\begin{figure}[H]
\centering    
\includegraphics[width=0.8\textwidth]{usecase}
\caption[Biểu đồ High level Usecase]{Biểu đồ High level Usecase }
\label{fig:usecase_diagram}
\end{figure}
Mô tả giản đồ Usecase
\begin{table}[H]
\centering
\caption{Bảng mô tả giản đồ Usecase của Web Server}
\label{table:usewebsite}
\begin{tabular}{|l|l|L{10cm}|}
\hline
STT & Tên            & Miêu tả                                                                                            \\ \hline
1   & View map       & Người dùng thấy được các node đang chạy trên google maps                                           \\ \hline
2   & View Statistic & Hiển thị các thống kê số lượng node hoạt động và số lượng dữ liệu thu thập                                                       \\ \hline
3   & View Graph     & Đồ thị hoạt động của Node
\\ \hline
4   & Contact        & Phản hồi ý kiến người dùng qua Gmail                                                               \\ \hline
5   & Add Node       & Thêm mới Node vào hoạt động                                                                        \\ \hline
6   & Update Node    & Thay đổi thông tin của Node( Lat, Lng, Phone)                                                      \\ \hline
7   & Replace Node   & Thay đổi 1 Node xảy ra sự cố bằng 1 Node khác                                                      \\ \hline
\end{tabular}
\end{table}




%%%%%%%%%%%%%%%%%%%%%%%%%%%%%%%%%%%%%%%%%%%%%%%%%%%%%%%%%%%%%%%%%%%%%%%%%%
% 				VẼ MẤY CÁI D.	Acitivity diagram VÀO ĐÂY %%%%%%%%%%%%%%%%%
%%%%%%%%%%%%%%%%%%%%%%%%%%%%%%%%%%%%%%%%%%%%%%%%%%%%%%%%%%%%%%%%%%%%%%%%%%%%%





\subsection*{Biểu đồ High level Usecase của ứng dụng di động}

\begin{figure}[H]
\centering    
\includegraphics[width=0.5\textwidth]{musecase}
\caption[Giản đồ High level Usecase của ứng dụng di động]{Giản đồ High level Usecase của ứng dụng di động}
\label{fig:app_usecase}
\end{figure}

Mô tả Usecase

\begin{table}[H]
\centering
\caption{Bảng mô tả giản đồ Usecase của ứng dụng di động}
\label{table:usecase_mobile}
\begin{tabular}{|l|l|L{10cm}|}
\hline
STT & Tên Use-case     & Mô tả                                                            \\ \hline
1   & View Node Info   & Thông tin của Node( Lat, lng, Phone, ID)                         \\ \hline
2   & View Graph       & 2 dạng đồ thị theo ngày (line chart, bar chart của các thông số) \\ \hline
3   & View Statistic & Hiển thị các thống kê số lượng node hoạt động và số lượng dữ liệu thu thập                                        \\ \hline
\end{tabular}
\end{table}







